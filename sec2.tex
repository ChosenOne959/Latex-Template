\fancyhead[LH]{寒武纪业务研究}
\rhead{\includegraphics[scale=0.25]{fig/particle_future.png}}  %在此处插入logo.pdf图片 图片靠左

\section{寒武纪业务研究}
\quad \quad 公司的主营业务是应用于各类云服务器、边缘计算设备、终端设备中人工智能核心
芯片的研发、设计和销售,为客户提供丰富的芯片产品与系统软件解决方案。公司的主
要产品包括终端智能处理器 IP、云端智能芯片及加速卡、边缘智能芯片及加速卡以及
与上述产品配套的基础系统软件平台。
\subsection{产业领域}
寒武纪已推出的产品体系覆盖了云端、边缘端的智能芯片及其加速卡、终端智能处理器IP,可满足云、边、端不同规模的人工智能计算需求。
公司的智能芯片和处理器产品可高效支持机器视觉(图像和视频的智能处理)、语音处理(语音识别与合成)、自然语言处理以及推荐系统等多样化的人工智能任务,
高效支持视觉、语音和自然语言处理等技术相互协作融合的多模态人工智能任务,辐射智慧互联网、智能制造、智能交通、智能教育、智慧金融、智能家居、智慧医疗等“智能+”产业。\par
同时,公司为云边端智能芯片和处理器产品研发了统一的基础系统软件平台,彻底打破云端、边缘端、终端之间的开发壁垒,无须繁琐的移植即可让同一人工智能应用程序便捷高效地
运行在公司云边端所有产品之上。云边端体系化的智能芯片和处理器产品以及完全统一的基础系统软件平台可大幅加速人工智能应用在各场景的落地,加快公司生态的拓展。
公司凭借领先的研发能力、可靠的产品质量和优秀的客户服务水平,在国内外积累了良好的品牌认知和优质的客户资源。目前公司产品广泛服务于知名芯片设计公司、服务器厂商
和产业公司,辐射互联网、云计算、能源、教育、金融、电信、交通、医疗等。
\subsection{公司经营模式}

从产业模式来看,集成电路企业主要包括IDM(垂直整合制造)、Fabless(无晶圆
厂)、Foundry(代工厂)以及封装测试企业(OSAT),集成电路设计行业运营模式主
要为其中的IDM模式和Fabless模式。\par
采用IDM模式运营的企业,其业务涵盖了从芯片设计、晶圆制造到封装测试整个流
程,能够发挥各个流程的协同效应,行业发展早期的大部分集成电路企业均采用该模式。
由于这样的垂直整合制造模式对企业从研发水平、生产管理能力、资金规模到业务量均
有很高的要求,目前仅有英特尔、三星、德州仪器等国际集成电路巨头采用这一运营模
式。\par
采用Fabless模式运营的企业,主要专注于芯片设计和产品销售两个环节,晶圆制造
和封装测试等流程均采用委外合作的方式进行。Fabless模式无需进行大量固定资产投
资,具有灵活性强、研发和技术导向、对市场需求反应迅速等特点,在集成电路行业日
益成熟、日趋专业化的背景下,成为目前集成电路设计企业的主要运营模式,英伟达、
高通、华为海思等领先集成电路设计企业均采用此模式。\par
公司自成立以来的经营模式均为Fabless模式,未曾发生变化,并将长期持续。

\subsection{主要产品及技术特点}
寒武纪自成立至今,一直专注于人工智能芯片设计领域,积累了较强的技术和研发优
势。寒武纪是目前国际上少数几家全面系统掌握了智能芯片及其基础系统软件研发和产品
化核心技术的企业之一,能提供云边端一体、软硬件协同、训练推理融合、具备统一生
态的系列化智能芯片产品和平台化基础系统软件,其具体产品如图\ref{products}。寒武纪掌握的智能处理器指令集、智能
处理器微架构、智能芯片编程语言、智能芯片高性能数学库等核心技术,具有壁垒高、
研发难、应用广等特点,对集成电路行业与人工智能产业具有重要的技术价值、经济价
值和生态价值。

\begin{figure}[!htb]
  \centering
  \includegraphics[width=0.9\linewidth]{fig/products.png}
  \caption{寒武纪产品汇总}
  \label{products}
\end{figure}
% \begin{table}[htb]
%     \centering
%       \caption{寒武纪产品汇总}
%       \begin{tabular}{|c|c|c|c|}
%     \hline
%     产品线 & 产品类型&主要产品 & 推出时间 \\
%     \hline
%     \multirow{6}*{云端}
%     &\multirow{4}*{云端智能芯片及加速卡}
%     &思元100(MLU100)芯片\newline 及云端智能加速卡
%     &2018年\\
%     \cline{3-4}
%     &&思元 270(MLU270)芯片\newline 及云端智
%     能加速卡&2019年\\
%     \cline{3-4}
%     &&思元 290(MLU290)芯片\newline 及云端智
%     能加速卡&2020年\\
%     \cline{3-4}
%     &&思元 370(MLU370)芯片\newline 及云端智
%     能加速卡&2021年、2022年\\
%     \cline{2-4}
    % &\multirow{2}*{训练整机}
    % &&玄思 1000 智能加速器 & 2020年\\
    % \cline{3-4}
    % &&玄思 1001 智能加速器&2022年\\
    % \cline{1-4}
    % 边缘端&边缘智能芯片及加速卡&思元 220(MLU220)芯片及边缘智
    % 能加速卡&2019年\\
    % \cline{1-4}
    % \multirow{4}*{IP授权及软件}
    % &\multirow{3}{终端智能处理器IP}
    % &寒武纪1A处理器&2016年\\
    % \cline{3-4}
    % &寒武纪 1H 处理器&2017年\\
    % \cline{3-4}
    % &寒武纪1M处理器&2018年\\
    % \cline{2-4}
    % &基础系统软件平台
    % &寒武纪基础软件开发平台(适用于公
    % 司所有芯片与处理器产品)&持续研发和
    % 升级,以适
    % 配新的芯片\\
    % \cline{1-4}
    % \end{tabular}
    % \label{table1}
    % \end{table}
\subsubsection{智能加速卡}
目前加速卡主要包含思元270系列,思元290系列,思元370系列,主要用于云端服务器加速。\par
A.\textbf{思元270(云端推理芯片)系列}:思元270集成了寒武纪在处理器架构领域的一系列创新性技术,处理非稀疏人工智能模型的理论峰值性能提升至上一代思元100的4倍,
达到128TOPS(INT8);同时兼容INT4和INT16运算,理论峰值分别达到256TOPS和64TOPS,具体对比如图\ref{siyuan100_siyuan270};支持浮点运算和混合精度运算。思元270的\textbf{核心优势在于}:
\begin{itemize}
  \item 思元270采用了寒武纪MLUv02架构,
  MLUv02架构不是简单的从上一代升级而来,新架构基于片上网络(NOC)构建,保证思元270芯片内多达16 个张量核心的并行效率。基于硬件的片内数据压缩,提升缓存有效容量和带宽。
  \item 新架构在采用INT8精度进行AI推理计算时,非稀疏网络性能比第一代加速卡提升高达4倍,可为系统提供40倍于CPU的超高能效比。
  \item 思元270芯片支持多类人工智能应用,寒武纪基础软件平台可以轻松部署推理环境。BANG Lang.编程环境可对计算资源做直接定制,满足多样化AI定制要求,专业而不专用。
\end{itemize}

\begin{figure}[!htb]
  \centering
  \includegraphics[width=0.9\linewidth]{fig/siyuan100_siyuan270.png}
  \caption{思元100与思元270理论峰值性能对比}
  \label{cambricon_software_framework}
\end{figure}

B.\textbf{思元290(云端训练)系列}:寒武纪思元290芯片,采用创新性的MLUv02扩展架构,使用台积电7nm先进制程工艺制造,在一颗芯片上集成了高达460亿的晶体管。
芯片具备多项关键性技术创新, MLU-Link™多芯互联技术,提供高带宽多链接的互连解决方案;HBM2内存提供AI训练中所需的高内存带宽;
vMLU帮助客户实现云端虚拟化及容器级的资源隔离。多种全新技术帮助AI计算应对性能、效率、扩展性、可靠性等多样化的挑战,相比上一代思元270性能又有了重大提升,
具体对比如图\ref{siyuan270_siyuan290}。思元290其\textbf{核心优势在于}:
\begin{itemize}
  \item 思元290基于MLUv02架构进行了多项扩展,实现峰值算力提升4倍、缓存带宽提高12倍、芯片间通讯带宽提高19倍。新架构采用7nm制程,可提供更高性能功耗比,以及多MLU系统的扩展能力。
  \item MLU-Link™多芯互联技术,首发于寒武纪思元290芯片,总带宽高达600GB/s,支持思元芯片间互联和跨系统互联,可实现纵向扩展,满足AI模型训练的需要。
  \item 寒武纪虚拟化技术vMLU,支持在思元290上实现4个相互隔离的AI计算实例,每个实例独占计算、内存和编解码资源,在虚拟化环境下仍可保持不低于$90\%$的极高效率,帮助客户充分利用硬件资源。
  \item 寒武纪基础软件平台采用端云一体架构,支持寒武纪全系列产品共享同样的软件接口和完备生态,可方便地进行AI应用的开发,迁移和调优,轻松实现云端开发训练模型,终端部署应用。
  \item 思元290采用寒武纪自适应精度训练方法。自适应精度训练可自适应调整人工智能模型不同层、不同数据类型的量化参数,同时量化参数调整周期也是自适应的,可在保证精度要求的基础上提高能效比。 
  \item 思元290承载了32G高带宽内存(HBM2),单芯片内存带宽高达1.23TB/秒,是思元270芯片的 12倍,有效解决传统加速器芯片内存带宽瓶颈问题,为用户提供更高的模型训练速度。
\end{itemize}

\begin{figure}[!htb]
  \centering
  \includegraphics[width=0.9\linewidth]{fig/siyuan270_siyuan290.png}
  \caption{思元270与思元290理论峰值性能对比}
  \label{siyuan270_siyuan290}
\end{figure}

C.\textbf{思元370(推理训练一体芯片)系列}:基于7nm制程工艺,思元370是寒武纪首款采用chiplet(芯粒)技术的AI芯片,集成了390亿个晶体管,最大算力高达256TOPS(INT8),是寒武纪第二代产品思元270算力的2倍。
凭借寒武纪最新智能芯片架构MLUarch03,思元370实测性能表现更为优秀。思元370也是国内第一款公开发布支持LPDDR5内存的云端AI芯片,内存带宽是上一代产品的3倍,
访存能效达GDDR6的1.5倍。搭载MLU-Link™多芯互联技术,在分布式训练或推理任务中为多颗思元370芯片提供高效协同能力。全新升级的寒武纪基础软件平台,
新增推理加速引擎MagicMind,实现训推一体,大幅提升了开发部署的效率,降低用户的学习成本、开发成本和运营成本,思元370的性能表现十分优秀,与业内主流GPU对比
展现出强大的性能与高效的能耗,具体对比如图\ref{siyuan370}。思元370的\textbf{核心优势在于}:
\begin{itemize}
  \item 寒武纪首次采用chiplet技术将2颗AI计算芯粒封装为一颗AI芯片,通过不同芯粒组合规格多样化的产品,为用户提供适用不同场景的高性价比AI芯片。
  \item 采用MLUarch03芯片架构,新一代张量运算单元,内置Supercharger模块大幅提升各类卷积效率;采用全新的多算子硬件融合技术,在软件融合的基础上大幅减少算子执行时间。
  \item 思元370芯片在业内率先支持LPDDR5内存,高带宽且低功耗,内存带宽是上一代产品的3倍,访存能效达GDDR6的1.5倍,可在板卡有限的功耗范围内给AI芯片分配更多的能源,输出更强大的算力。
\end{itemize}

\begin{figure}[!htb]
  \centering
  \includegraphics[width=0.9\linewidth]{fig/siyuan370.png}
  \caption{思元370系列板卡与业内主流GPU性能对比}
  \label{siyuan370}
\end{figure}

\subsubsection{智能加速系统}
寒武纪玄思1000智能加速器整机在2U机箱内集成了4颗思元290智能芯片,可实现AI算力多向扩展,满足性能、扩展性、灵活性、鲁棒性的要求。
\subsubsection{智能边缘计算模组}
思元220边缘计算模组,MLU220是一款专门用于边缘计算应用场景的AI加速产品(边缘人工智能加速卡)。
产品集成4核ARM CORTEX A55,LPDDR4x内存及丰富的外围接口。用户既可以使用MLU220作为AI加速协处理器,也可以使用其实现SOC方案。
其核心优势是:
\begin{itemize}
  \item 采用了寒武纪MLUv02架构。
  \item 面向边缘侧量身定制的只能解决方案,拥有较小的体积。
  \item 思元220芯片支持多类网络模型,寒武纪基础软件平台可以轻松部署推理环境。BANG Lang.编程环境可对计算资源做直接定制,满足多样化AI定制要求。
\end{itemize}
\subsubsection{终端智能处理器IP}
目前寒武纪的智能终端IP主要是Cambricon-1M系列和Cambricon-1H系列。
Cambricon-1H系列作为寒武纪第二代架构,较初代产品其能效比有着数倍提升,可以广泛应用于计算机视觉、语音识别、自然语言处理等人工智能处理关键领域。
Cambricon-1M系列作为寒武纪第三代架构,具备了更优性能、更低功耗和更强的完备性,混合支持fp32/ fp16 /int32 /int16 /int8 /int4位宽,增加了压缩解压缩模块。 
在上代产品的基础上,可支持个性化人工智能应用,也可使用于多路视频实时处理和自动驾驶等领域。
\subsubsection{软件开发平台}
\begin{figure}[!htb]
  \centering
  \includegraphics[width=0.9\linewidth]{fig/cambricon_software_framework.png}
  \caption{寒武纪基础软件平台}
  \label{cambricon_software_framework}
\end{figure}
寒武纪的软件开发平台主要包含两个部分的内容,一是AI算法框架,二是相关开发工具,
整体如图\ref{cambricon_software_framework},根据应用不同可以将其分为训练和推理两个部分。\par
其中训练平台架构如图\ref{training_framework},该训练软件平台支持基于主流开源框架原生分布
式通信方式,同时也支持Horovod开源分布式通信框架,可实现从单卡到集群的分布式训练任务。
支持多种网络拓扑组织方式,并完整支持数据并行、模型并行和混合并行的训练方法。
训练软件平台支持丰富的图形图像、语音、推荐以及NLP训练任务。
通过底层算子库CNNL和通信库CNCL,在实际训练业务中达到业界领先的硬件计算效率和通信效率。
同时提供模型快速迁移方法,帮助用户快速完成现有业务模型的迁移。\par
推理平台的架构如图\ref{inference_framework},其核心在于MagicMind推理加速引擎。
MagicMind是寒武纪全新打造的推理加速引擎,也是业界首个基于MLIR图编译技术达到商业化部署能力的推理引擎。
借助MagicMind,用户仅需投入极少的开发成本,即可将推理业务部署到寒武纪全系列产品上,并获得颇具竞争力的性能。\par
寒武纪为云边端智能芯片与处理器产品提供统一的基础系统软件平台,并通过持续优化和迭代,以适配新的芯片。
寒武纪的基础系统软件平台打破了不同场景之间的软件开发壁垒,兼具灵活性和可扩展性的优势,无须繁琐的移植即可让同一人工智能应用程序便捷高效地运行在公司云边端系列化芯片与处理器产品之上。
\begin{figure}[htb]
  \centering
\begin{minipage}{0.49\linewidth}
  \centering
  \includegraphics[width=0.8\linewidth]{fig/training_framework.png}
  \caption{寒武纪训练平台}\label{training_framework}
\end{minipage}
\begin{minipage}{0.49\linewidth}
  \centering
  \includegraphics[width=0.8\linewidth]{fig/inference_framework.png}
  \caption{寒武纪推理平台}\label{inference_framework}
\end{minipage}
\end{figure}





   

% \begin{figure}[htb]
%     \centering
%   \begin{minipage}{0.49\linewidth}
%     \centering
%     \includegraphics[width=0.8\linewidth]{fig/FPGA_proportion.png}
%     \caption{2021年全球FPGA市场格局}\label{FPGA_proportion}
%   \end{minipage}
%   \begin{minipage}{0.47\linewidth}
%     \centering
%     \includegraphics[width=0.8\linewidth]{fig/China_FPGA.png}
%     \caption{2021年中国FPGA市场格局}\label{China_FPGA}
%   \end{minipage}
% \end{figure}

