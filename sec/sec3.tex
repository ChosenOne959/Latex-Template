
\fancyhead[LH]{寒武纪财务分析}
\rhead{\includegraphics[scale=0.25]{fig/particle_future.png}}  %在此处插入logo.pdf图片 图片靠左

\section{寒武纪财务分析}
\subsection{业务营收情况}
据寒武纪公司2022年审计报告显示,
寒武纪公司的营业收入主要来自于智能计算集群系统销售业务、云端产品线
销售业务和边缘端产品线销售业务。2022 年度,寒武纪公司营业收入金额为人
民币 72,903.46 万元,其中智能计算集群系统销售业务的营业收入为人民币
45,851.03 万元,占营业收入的$62.89\%$,云端产品线销售业务的营业收入为人民
币 21,944.89 万元,占营业收入的$30.10\%$,边缘端产品线销售业务的营业收入
为人民币 3,783.77 万元,占营业收入的$5.19\%$,具体如图\ref{2022_revenue}所示。\par
据浙商证券2023年3月18日发布的寒武纪深度报告显示,寒武纪公司在2021年边缘产品线上业务收入
约为1.75亿元,同比增长$741.10\%$,在云端产品线上的业务收入约为0.80亿元,同比减少了$6.98\%$,
在智能计算集群系统业务的营收约为4.56亿元,同比增长$39.91$,具体如图\ref{2021_revenue}。
\begin{figure}[htb]
    \centering
  \begin{minipage}{0.49\linewidth}
    \centering
    \includegraphics[width=0.8\linewidth]{fig/2022_revenue.png}
    \caption{寒武纪2022年业务营收占比}\label{2022_revenue}
  \end{minipage}
  \begin{minipage}{0.49\linewidth}
    \centering
    \includegraphics[width=0.8\linewidth]{fig/2021_revenue.png}
    \caption{寒武纪2021年业务营收占比}\label{2021_revenue}
  \end{minipage}
  \end{figure}
  
\subsection{盈利情况}
由choice终端提供的企业产品线毛利率表格,我们可以看到寒武纪公司各业务线的大致毛利率情况,其结果如图\ref{gross_profit_rate}。可见IP销售毛利最高,但由于
华为与其在2017年终止合作,导致IP销售在寒武纪内部的营收份额骤然下降,如今已经不能够称其为主营业务,这也导致了寒武纪的整体毛利率大幅下降。
2017 年度、2018 年度及 2019 年度,寒武纪综合毛利率分别为$99.96\%$、$99.90\%$及
$68.19\%$,主要原因系 2019 年公司拓展了云端智能芯片及加速卡、智能计算集群系统业
务,导致公司毛利率下降,并进而影响公司的盈利能力及业绩表现。
\begin{figure}[!htb]
    \centering
    \includegraphics[width=0.9\linewidth]{fig/gross_profit_rate.png}
    \caption{寒武纪各业务线毛利率}
    \label{gross_profit_rate}
  \end{figure}
根据寒武纪公司发布的2022年度报告以及2023年第一季度报告,可以得知:寒武纪公司\textbf{尚未实现盈利且存在累计未弥补亏损},
寒武纪为确保“云边端”各系列智能芯片产品及基础系统软件平台的高质量迭代,在竞争激烈的市场中保持技术领
先优势,持续加大研发投入。寒武纪公司发布的2022年度报告中显示,2020年寒武纪公司在研发投入占营业收入的比例为$167.41\%$,
2021年研发投入占营业收入的比例为$157.51\%$,2022年研发投入占营业收入的比例为$208.92\%$,由此可见寒武纪十分重视研发投入,注重
提升自身技术,其产品迭代速度也处于业内较高水平,但正是巨大的研发投入,导致其至今未能实现盈利。同时从图\ref{gross_profit_rate}中可以
看出,边缘端业务毛利明显低于寒武纪其它业务,这应该与边缘端芯片研发过程中高昂的测试成本有关。根据寒武纪公司发布的2022年度报告营收成本部分的分析,
如图\ref{operating_cost}
\begin{figure}[!htb]
  \centering
  \includegraphics[width=0.9\linewidth]{fig/operating_cost.png}
  \caption{寒武纪边缘端与云端营业成本分析对比}
  \label{operating_cost}
\end{figure}
可以得知,寒武纪边缘端芯片的测试成本高达$4,009,984.26$,远高于云端产品线的封装测试成本。




