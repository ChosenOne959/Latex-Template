\fancyhead[LH]{左边的页眉}
\rhead{\includegraphics[scale=0.25]{fig/particle_future.png}}  %在此处插入logo.pdf图片 图片靠左

\section{一级子标题}
% 图片文字环绕排布
% \begin{wrapfigure}{l}{3cm}%靠文字内容的右侧
%     \centering
%     \includegraphics[width=0.22\textwidth]{fig/chenTianShi.png}
%     \caption{\footnotesize 陈天石}
%     \end{wrapfigure}
%     陈天石出自“中科大少年班”,师从陈国良院士与姚新教授
%     2010年毕业于中国科学技术大学计算机学院,获工学博士学位。同年进入中国科学院计算技术研究所工作。研究方向为计算机体系结构和计算智能。
%     在IEEE/ACM Transactions、Theoretical Computer Science、ASPLOS、ISCA、MICRO、HPCA、IJCAI、AAAI、SPAA、DATE等重要期刊和会议上发表论文40余篇。
%     曾获ASPLOS最佳论文奖(2014)、MICRO最佳论文奖(2014)、全国百篇优秀博士论文提名奖(2012)、中国计算机学会优秀博士论文奖(2011)、中国科学院优秀博
%     士论文奖(2011)、中国科学院院长奖(2010)等荣誉,并入选计算所百星计划(2011)、CCF-Intel青年学者提升计划(2014)。
%     作为寒武纪的“领舵人”,陈天石在科研方面的实力,对于产业发展所储备的丰富经验能够很好的带领团队,拨开迷雾。

\subsection{二级子标题}
%图片并排
% \begin{figure}[htb]
%     \centering
%   \begin{minipage}{0.49\linewidth}
%     \centering
%     \includegraphics[width=0.8\linewidth]{fig/staff_composition.png}
%     \caption{寒武纪人员专业构成}\label{staff_composition}
%   \end{minipage}
%   \begin{minipage}{0.49\linewidth}
%     \centering
%     \includegraphics[width=0.8\linewidth]{fig/education_background.png}
%     \caption{寒武纪人员学历构成}\label{education_background}
%   \end{minipage}
% \end{figure}
