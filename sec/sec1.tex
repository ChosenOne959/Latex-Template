\fancyhead[LH]{寒武纪公司简介}
\rhead{\includegraphics[scale=0.25]{fig/particle_future.png}}  %在此处插入logo.pdf图片 图片靠左

\section{寒武纪公司简介}
\quad \quad 寒武纪,机构全称中科寒武纪科技股份有限公司,英文名称Cambricon Technologies Corporation Limited
,成立日期为$2016-03-15$,上市日期为$2020-07-20$,专注于人工智能芯片产品的研发与技术创新,致力于打造人工智能领域的核心处理器芯片,让机器更好地理解和服务人类。
寒武纪提供云边端一体、软硬件协同、训练推理融合、具备统一生态的系列化智能芯片产品和平台化基础系统软件。寒武纪产品广泛应用于服务器厂商和产业公司,面向互联网、
金融、交通、能源、电力和制造等领域的复杂AI应用场景提供充裕算力,推动人工智能赋能产业升级。

\subsection{创业团队}

\subsubsection{创始人}
\begin{wrapfigure}{l}{3cm}%靠文字内容的右侧
    \centering
    \includegraphics[width=0.22\textwidth]{fig/chenTianShi.png}
    \caption{\footnotesize 陈天石}
    \end{wrapfigure}
    陈天石出自“中科大少年班”,师从陈国良院士与姚新教授
    2010年毕业于中国科学技术大学计算机学院,获工学博士学位。同年进入中国科学院计算技术研究所工作。研究方向为计算机体系结构和计算智能。
    在IEEE/ACM Transactions、Theoretical Computer Science、ASPLOS、ISCA、MICRO、HPCA、IJCAI、AAAI、SPAA、DATE等重要期刊和会议上发表论文40余篇。
    曾获ASPLOS最佳论文奖(2014)、MICRO最佳论文奖(2014)、全国百篇优秀博士论文提名奖(2012)、中国计算机学会优秀博士论文奖(2011)、中国科学院优秀博
    士论文奖(2011)、中国科学院院长奖(2010)等荣誉,并入选计算所百星计划(2011)、CCF-Intel青年学者提升计划(2014)。
    作为寒武纪的“领舵人”,陈天石在科研方面的实力,对于产业发展所储备的丰富经验能够很好的带领团队,拨开迷雾。

\begin{wrapfigure}{l}{3cm}%靠文字内容的右侧
    \centering
    \includegraphics[width=0.2\textwidth]{fig/chenYunJi.png}
    \caption{\footnotesize 陈云霁}
\end{wrapfigure}
陈云霁中国科学院计算技术研究所副所长,研究员,博士生导师,研究方向为计算机体系结构、机器学习。
他带领智能处理器研究中心,研制了国际上首个深度学习专用处理器芯片。
相关技术广泛应用于多种手机和服务器产品中。在项目经历方面,
陈云霁于2002年加入中科院计算所,跟随胡伟武研究员硕博连读,成为当时龙芯研发团队中最年轻的成员。
博士毕业后,他留在了计算所。25岁时,\textbf{陈云霁成为8核龙芯3号的主架构师}。学术成就方面,
他在ISCA、HPCA、ASPLOS、MICRO、ICSE、ISSCC、Hot Chips、IJCAI、IEEE TC、IEEE TPDS、IEEE TCAD、ACM TOCS和CACM
等学术会议及期刊上发表论文100余篇,申请专利100余项,获得了ASPLOS和MICRO等计算机体系结构顶级国际会议最佳论文奖(亚洲迄今仅有的两次)。
哈佛、斯坦福、麻省理工、谷歌等上百个国际知名机构跟踪引用他的学术论文开展深度学习处理器研究。因此,他被Science杂志评价为该方向的“先驱”和“领导者”  。
曾获国家杰出青年科学基金、中国青年科技奖、国家自然科学基金“优秀青年基金”、国家万人计划“青年拔尖人才”、中科院青年科学家奖和中国计算机学会青年科学家奖,
并被MIT技术评论评为全球35位杰出青年创新者(2015年度)。他还两次获得中国科学院优秀导师奖。\par
两兄弟在寒武纪的角色各有侧重不同,陈云霁在公司职务上更偏研究,思考技术路径相关的部分,曾在寒武纪中任职首席科学家,但2016年底他卸任了寒武纪首席科学家一职
并不再担任寒武纪任何职务,而
陈天石不仅要做科研也需要掌管公司事务,并以寒武纪创始人、CEO的身份出现在公众视野。
\subsubsection{团队背景}
根据寒武纪招股书披露,公司的核心技术人员如下:
\begin{itemize}
    \item 梁军:梁军在2017年加入寒武纪担任CTO一职。供职于寒武纪期间,梁军参与研究并申请发明专利138项、PCT10项,
    加入寒武纪之前,梁军先后就职于华为技术有限公司、 深圳市海思半导体有限公司,就在加入寒武纪并担任CTO后的第一年,寒武纪就与梁军前东家华为,在手机终端芯片业务上展开了IP授权合作。
    梁军于2022年2月10日通知公司解除劳动合同。在官宣次日,寒武纪股价暴跌超过$18\%$,上交所也对寒武纪下发监管工作函。梁军在知乎上风评较差。
    \item 刘少礼:刘少礼博士分别于南开大学、中国科学院获得理学学士和工学博士学位。于2016年3月起出任寒武纪副总裁,并长期担任公司董事。
    刘少礼博士是寒武纪1A(全球首款商用终端智能处理器)、寒武纪1H和寒武纪1M等多款处理器产品的主架构师,也是寒武纪思元系列云端智能芯片产品的主要研发领导者之一。
    同时,刘少礼博士作为第一作者提出了寒武纪Cambricon指令集,该指令集已成为全球首款公开发布的智能芯片指令集。
    \item 刘道福:刘道福,2006年考入中国科学技术大学计算机科学技术系,2015年获得中国科学院计算技术研究所工学博士 ,2015年至2019年就职于中科院计算所,历任助理研究员、
    高级工程师,2016年作为公司创始团队成员加入公司,曾任公司副总经理,现已离任。 
    \item 王在:2002年考入中国科学技术大学,2006年7月于中国科学技术大学计算机学院获学士学位,2011年7月于中国科学技术大学获博士学位,2011年至2015年就职于
    郑州商品交易所并任核心交易系统工程师,2015年至2016年就职于中原银行并任信息科技部电子银行系统主管,2016年至2018年就职于中科院计算所从事科研工作。2016年作为公司创始团队成员加入公司,现任公司董事、副总经理兼首席运营官。 
\end{itemize}
\subsection{公司组织架构}
\subsubsection{公司人员架构}
根据choice金融终端寒武纪公司深度资料中员工构成数据,截至2022年12月31日,寒武纪员工中有$79.49\%$为研发人员,员工专业具体占比如图\ref{staff_composition},
且公司人员中有$63.72\%$拥有硕士及以上学位,员工具体学历构成如图\ref{education_background}。
同时,寒武纪的核心创始团队都具备多年人工智能芯片领域研发
和设计经验,其中不乏人工智能芯片领域的顶级大咖。
\begin{figure}[htb]
    \centering
  \begin{minipage}{0.49\linewidth}
    \centering
    \includegraphics[width=0.8\linewidth]{fig/staff_composition.png}
    \caption{寒武纪人员专业构成}\label{staff_composition}
  \end{minipage}
  \begin{minipage}{0.49\linewidth}
    \centering
    \includegraphics[width=0.8\linewidth]{fig/education_background.png}
    \caption{寒武纪人员学历构成}\label{education_background}
  \end{minipage}
\end{figure}
\subsubsection{股权分配}
寒武纪最大股东是陈天石,个人持股约$28.83\%$,其余前十大的股东为北京中科算源资产管理有限公司,占股$15.84$、北京艾溪科技中心(执行事务合伙人为陈天石),占股
$7.39\%$、国投(上海)创业投资管理有限公司,占股$3.1\%$、招商银行股份有限公司,占股$2.53\%$,
苏州工业园区古生代创业投资企业,占股$2.33\%$、北京艾加溪科技中心,占股$2.05\%$、杭州阿里创业投资有限公司,占股$1.68\%$
、河南国新启迪基金管理有限公司,占股$1.40\%$、北京纳远明志信息技术咨询有限公司,占股$1.1\%$,十大股东
占股份额变化详见图 \ref{equity_distribution}。
\begin{figure}[h]
    \centering
    \includegraphics[width=0.9\linewidth]{fig/equity_distribution.png}
    \caption{寒武纪前十大股东持股比例}\label{equity_distribution}
\end{figure}

在高管持股情况方面,陈天石作为公司董事长、总经理,占高管持股比例$99.93\%$,
叶昊尹作为总经理,占高管持股比例$0.03\%$,刘少礼作为副总经理,占高管持股比例$0.01\%$,
王在作为副总经理,占高管持股比例$0.01\%$,陈煜作为副总经理,占高管持股比例$0.01\%$。
总体上看公司的股权结构稳定,深度联合国资和战略投资者,陈天石本人也对公司有稳定的掌控权。
\subsection{公司发展历程}
\begin{figure}[h]
    \centering
    \includegraphics[width=1.0\linewidth]{fig/cambricon_development.png}
    \caption{寒武纪发展史}\label{cambricon_development}
\end{figure}
\subsubsection{重要发展节点}
\begin{itemize}
    \item 2012年陈氏两兄弟带头启动了神经网络处理器(AI芯片项目)DianNao。
    \item 2015 年陈氏兄弟主导的世界首款深度学习专用处理器原型芯片“寒武纪”首次成功流片。
    \item 2016年寒武纪成立,拥有十几年人工智能基础学术研究的寒武纪团队正式出道,行事谨慎的陈天石担任公司 CEO,而陈云霁选择继续在计算所搞科研,为寒武纪的首席科学家。寒武纪推出首款产品。2016年寒武纪推出深度神经网络处理器-1A(Cambricon-1A)是世界首款商用深度学习专用处理器。与特斯拉增强型自动辅助驾驶、
    IBM Watson等同时入选第三届世界互联网大会(乌镇)评选的十五项“世界互联网领先科技成果”。
    \item 2016年中科曙光与寒武纪科技有限公司达成战略合作协议。中科曙光公司相关负责人表示,中科曙光致力于打造数据中国,
    而携手寒武纪科技不仅有利于打通数据中国战略中的技术闭环、完善产业生态,还将为中科曙光在未来智能时代中的布局奠定基础。
    2019年,寒武纪云端智能芯片及加速卡向中科曙光销售额为6384.43,关联销售占比约为$80.94\%$。
    \item 2017年与华为海思达成合作。华为海思半导体手机处理器麒麟970、980采用寒武纪科技的人工智能副单元。2017年至2019年,公司终端智能处理器 IP 授权业务收入分别为 771.27 万
    元、11,666.21 万元和 6,877.12 万元,占主营业务收入的比例分别为$98.95\%$、$99.69\%$和
    $15.49\%$。其中,本公司对华为海思终端智能处理器IP授权业务的销售金额为 771.27万
    元、11,425.64 万元和 6,365.80 万元,占到公司终端智能处理器 IP 授权业务销售收入比
    例的$100.00\%$、$97.94\%$和$92.56\%$。
    \item 2018年推出思元100处理器,是寒武纪推出的第一款智能处理板卡产品,标志着寒武纪开始步入云端处理芯片领域。
    \item 2019年华为海思选择自研终端智能芯片,未与寒武纪继续合作。2019年终端智能处理器IP授权业务收入相较于2018
    年下滑$41.23\%$。
    \item 2019年推出思源220,寒武纪布局边缘智能计算,完成云边端产业全面部署。
    \item 2020年7月20日在上海证券交易所科创板挂牌上市。股东包括阿里创投、科大讯飞、湖北联想、中科图灵、国新资本、中科院创投等
\end{itemize}
\subsubsection{公司战略}
公司以“为客户创造价值,成为持续创新的智能时代领导者”为使命,以“让机器更好地理
解和服务人类”为愿景,聚焦于人工智能芯片领域,为客户提供系列化的人工智能芯片产品与技
术支持服务。未来公司将围绕自身的核心优势、提升核心技术,结合内外部资源,以自主创新为
驱动,不断推动企业发展,围绕人工智能核心驱动力——计算能力,坚持云边端一体化,坚持软
硬件协同,为智能云计算、智能边缘、智能终端、智能驾驶等场景提供芯片及加速卡产品,矢志
成为国际领先的人工智能芯片设计公司,服务全球客户。\par
鉴于集成电路设计行业是人才、技术和资金密集型的行业,行业的发展受研发、技术和管理
能力驱动,公司将密切关注中国及全球市场智能芯片需求,从产品定义、研发规划、资源整合、
委外合作以及产业链协同等方面制定发展战略,进一步提升公司的核心研发能力、产品设计能力
和市场地位,实现高速发展。
\subsubsection{重要融资历程}
根据天眼查提供的数据,寒武纪在上市前共经历7轮投资,其中A+轮投资方只有北京艾加溪科技中心(有限合伙),其执行合伙人为王在,目前仍就职于寒武纪,担任副总经理,
投资金额为1.48亿元,B+投资方为越秀产业基金,但具体金额未披露,其余重要轮次融资细节如图\ref{financing_history}
\begin{figure}[h]
    \centering
    \includegraphics[width=0.9\linewidth]{fig/financing_history.png}
    \caption{寒武纪重要融资历程}
    \label{financing_history}
\end{figure}

